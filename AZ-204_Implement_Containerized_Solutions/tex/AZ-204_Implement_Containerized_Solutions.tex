
\documentclass{article}
\usepackage{geometry}
\geometry{a4paper, margin=1in}
\usepackage{hyperref}
\usepackage{titlesec}
\usepackage{listings}
\usepackage{xcolor}

\titleformat{\section}{\large\bfseries}{\thesection}{1em}{}

\title{AZ-204 Study Guide: Implement Containerized Solutions}
\author{}
\date{}

\lstset{
    basicstyle=\ttfamily\footnotesize,
    keywordstyle=\color{blue},
    commentstyle=\color{green},
    stringstyle=\color{red},
    breaklines=true,
    showstringspaces=false,
    frame=single,
    backgroundcolor=\color{gray!10}
}

\begin{document}

\maketitle

\section{Overview of Containerized Solutions in Azure}
Containers package applications and their dependencies into a single, portable unit, enabling consistency across different environments. Azure offers multiple services for running containers, including Azure Kubernetes Service (AKS), Azure Container Instances (ACI), and Azure App Service for Containers.

\section{Azure Kubernetes Service (AKS)}
AKS is a managed Kubernetes service that simplifies deploying, managing, and scaling containerized applications. Provides features like auto-scaling, monitoring, and integrations with Azure DevOps and GitHub Actions for CI/CD.

\textbf{Azure CLI Commands:}
\begin{lstlisting}[language=bash]
# Create a Resource Group
az group create --name MyResourceGroup --location eastus

# Create an AKS Cluster
az aks create --resource-group MyResourceGroup --name MyAKSCluster --node-count 1 --enable-addons monitoring --generate-ssh-keys

# Get AKS Credentials to connect to the cluster
az aks get-credentials --resource-group MyResourceGroup --name MyAKSCluster
\end{lstlisting}

\textbf{Hands-On Tip:}
\begin{itemize}
    \item Deploy a sample application on AKS using \texttt{kubectl} commands to understand how to manage pods, deployments, and services.
\end{itemize}

\section{Azure Container Instances (ACI)}
ACI is a serverless container service that allows you to run containers without managing the underlying infrastructure. Ideal for quick deployment of containerized applications or for burst scenarios where scaling needs fluctuate.

\textbf{Azure CLI Commands:}
\begin{lstlisting}[language=bash]
# Create a Container Instance
az container create --resource-group MyResourceGroup --name mycontainer --image mcr.microsoft.com/azuredocs/aci-helloworld --dns-name-label mycontainerdemo --ports 80
\end{lstlisting}

\textbf{Hands-On Tip:}
\begin{itemize}
    \item Experiment with ACI by deploying simple applications and connecting them with other Azure services like Storage or Cosmos DB.
\end{itemize}

\section{Azure App Service for Containers}
Azure App Service can host Docker containers, providing a PaaS environment with built-in scaling, load balancing, and security features. Supports custom Docker images and multi-container deployments using Docker Compose.

\textbf{Azure CLI Commands:}
\begin{lstlisting}[language=bash]
# Create an App Service Plan
az appservice plan create --name MyAppServicePlan --resource-group MyResourceGroup --is-linux --sku B1

# Create a Web App for Containers
az webapp create --resource-group MyResourceGroup --plan MyAppServicePlan --name MyContainerApp --deployment-container-image-name mcr.microsoft.com/dotnet/samples:aspnetapp
\end{lstlisting}

\textbf{Hands-On Tip:}
\begin{itemize}
    \item Deploy a multi-container app using Docker Compose by uploading your docker-compose.yml file to Azure App Service.
\end{itemize}

\section{Container Registry (ACR)}
Azure Container Registry (ACR) is a managed Docker registry service that stores and manages container images for all types of deployments. Integrates with AKS, ACI, and other Azure services for seamless image deployments.

\textbf{Azure CLI Commands:}
\begin{lstlisting}[language=bash]
# Create a Container Registry
az acr create --resource-group MyResourceGroup --name MyContainerRegistry --sku Basic

# Log in to the ACR
az acr login --name MyContainerRegistry

# Push an image to the ACR
docker tag myimage MyContainerRegistry.azurecr.io/myimage:v1
docker push MyContainerRegistry.azurecr.io/myimage:v1
\end{lstlisting}

\section{CI/CD for Containers}
Automate the build, test, and deployment of containers using Azure Pipelines, GitHub Actions, or other CI/CD tools. Set up continuous deployment to automatically push updates to AKS, ACI, or Azure App Service when new images are built.

\textbf{Hands-On Tip:}
\begin{itemize}
    \item Set up a GitHub Actions workflow to build and push Docker images to Azure Container Registry, then deploy to AKS.
\end{itemize}

\section{Monitoring and Troubleshooting}
\begin{itemize}
    \item \textbf{Azure Monitor}: Integrate with Azure Monitor to collect metrics, logs, and set up alerts for containerized applications.
    \item \textbf{Log Analytics}: Use Log Analytics to analyze performance data and troubleshoot issues in containerized environments.
\end{itemize}

\textbf{Azure CLI Commands:}
\begin{lstlisting}[language=bash]
# Enable monitoring on AKS
az aks enable-addons --resource-group MyResourceGroup --name MyAKSCluster --addons monitoring
\end{lstlisting}

\textbf{Hands-On Tip:}
\begin{itemize}
    \item Use Azure Monitor's Container Insights to view metrics, logs, and performance data for your AKS clusters.
\end{itemize}

\section{Security for Containerized Solutions}
\begin{itemize}
    \item \textbf{Network Policies}: Use network policies to control traffic flow between pods in AKS.
    \item \textbf{Private Link}: Securely connect your containerized applications to other Azure services using Private Link.
    \item \textbf{Azure Key Vault}: Securely manage and access secrets, keys, and certificates from your containers.
\end{itemize}

\textbf{Hands-On Tip:}
\begin{itemize}
    \item Set up Azure Key Vault integration with your AKS cluster to securely store and access application secrets.
\end{itemize}

\end{document}
