
\documentclass{article}
\usepackage{geometry}
\geometry{a4paper, margin=1in}
\usepackage{hyperref}
\usepackage{titlesec}
\usepackage{listings}
\usepackage{xcolor}

\titleformat{\section}{\large\bfseries}{\thesection}{1em}{}

\title{AZ-204 Study Guide: Implement Azure Functions}
\author{}
\date{}

\lstset{
    basicstyle=\ttfamily\footnotesize,
    keywordstyle=\color{blue},
    commentstyle=\color{green},
    stringstyle=\color{red},
    breaklines=true,
    showstringspaces=false,
    frame=single,
    backgroundcolor=\color{gray!10}
}

\begin{document}

\maketitle

\section{Overview of Azure Functions}
Azure Functions is a serverless compute service that enables you to run event-driven code without managing infrastructure. It supports multiple programming languages, including C\#, JavaScript, Python, Java, and PowerShell. Ideal for microservices, data processing, and integration tasks.

\section{Creating an Azure Function App}
\begin{itemize}
    \item Use Azure Portal, Azure CLI, Visual Studio, or VS Code to create and deploy functions.
    \item Key components: Function App, Hosting Plan (Consumption, Premium, Dedicated), and Triggers (HTTP, Timer, Blob, etc.).
\end{itemize}

\textbf{Azure CLI Commands:}
\begin{lstlisting}[language=bash]
# Create a Resource Group
az group create --name MyResourceGroup --location eastus

# Create a Function App in Consumption Plan
az functionapp create --resource-group MyResourceGroup --os-type Windows --consumption-plan-location eastus --runtime dotnet --functions-version 4 --name MyUniqueFunctionAppName --storage-account mystorageaccount
\end{lstlisting}

\section{Triggers and Bindings}
\begin{itemize}
    \item \textbf{Triggers}: Determine how a function is invoked (HTTP, Timer, Blob, Queue, Event Grid).
    \item \textbf{Bindings}: Connect your function to other resources (input/output bindings like Cosmos DB, Blob Storage).
\end{itemize}

\textbf{Hands-On Tip:}
\begin{itemize}
    \item Create an HTTP-triggered function and test it locally using VS Code with the Azure Functions extension.
\end{itemize}

\section{Developing and Testing Functions}
\begin{itemize}
    \item Develop in local environments using Azure Functions Core Tools.
    \item Test using Postman, curl, or Azure Portal.
    \item Debug locally in VS Code or Visual Studio.
\end{itemize}

\textbf{Azure CLI Commands:}
\begin{lstlisting}[language=bash]
# Start Azure Functions Core Tools locally
func start
\end{lstlisting}

\section{Deployment Options}
\begin{itemize}
    \item Continuous integration using Azure DevOps, GitHub Actions, or other CI/CD tools.
    \item Manual deployment using Azure CLI, ZIP Deploy, or FTP.
\end{itemize}

\textbf{Hands-On Tip:}
\begin{itemize}
    \item Set up a GitHub Actions workflow to deploy your Azure Function App on each code push to the main branch.
\end{itemize}

\section{Scaling and Performance Management}
\begin{itemize}
    \item \textbf{Consumption Plan}: Automatically scales based on demand; pay only for resources consumed.
    \item \textbf{Premium Plan}: Scales automatically with pre-warmed instances for faster start times.
    \item \textbf{Dedicated Plan (App Service Plan)}: Fixed scaling, useful when running other App Services together.
\end{itemize}

\textbf{Azure CLI Commands:}
\begin{lstlisting}[language=bash]
# Update Function App to Premium Plan
az functionapp update --name MyUniqueFunctionAppName --resource-group MyResourceGroup --plan MyPremiumPlan
\end{lstlisting}

\section{Monitoring and Troubleshooting}
\begin{itemize}
    \item \textbf{Application Insights}: Integrated monitoring for performance, logging, and analytics.
    \item \textbf{Log Stream}: View real-time logs of running functions.
\end{itemize}

\textbf{Hands-On Tip:}
\begin{itemize}
    \item Enable Application Insights and monitor the performance of your Azure Functions in the portal.
\end{itemize}

\textbf{Azure CLI Commands:}
\begin{lstlisting}[language=bash]
# Enable Application Insights
az monitor app-insights component create --app MyUniqueFunctionAppName --location eastus --resource-group MyResourceGroup
\end{lstlisting}

\section{Security}
\begin{itemize}
    \item \textbf{Authentication/Authorization}: Use Azure AD, OAuth providers, or API keys to secure functions.
    \item \textbf{Managed Identity}: Access Azure resources securely without managing credentials.
\end{itemize}

\textbf{Hands-On Tip:}
\begin{itemize}
    \item Enable managed identity on your function app and use it to access Azure Key Vault securely.
\end{itemize}

\end{document}
