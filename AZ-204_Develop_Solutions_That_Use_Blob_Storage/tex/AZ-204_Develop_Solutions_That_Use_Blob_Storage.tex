
\documentclass{article}
\usepackage{geometry}
\geometry{a4paper, margin=1in}
\usepackage{hyperref}
\usepackage{titlesec}
\usepackage{listings}
\usepackage{xcolor}

\titleformat{\section}{\large\bfseries}{\thesection}{1em}{}

\title{AZ-204 Study Guide: Develop Solutions that Use Blob Storage}
\author{}
\date{}

\lstset{
    basicstyle=\ttfamily\footnotesize,
    keywordstyle=\color{blue},
    commentstyle=\color{green},
    stringstyle=\color{red},
    breaklines=true,
    showstringspaces=false,
    frame=single,
    backgroundcolor=\color{gray!10}
}

\begin{document}

\maketitle

\section{Overview of Azure Blob Storage}
Azure Blob Storage is a scalable object storage solution for storing unstructured data such as text, binary data, images, and video. Key scenarios include serving images or documents directly to a browser, streaming video and audio, or storing backups and archival data. Blob Storage supports different types: Block Blobs, Append Blobs, and Page Blobs.

\section{Blob Types}
\begin{itemize}
    \item \textbf{Block Blobs}: Ideal for storing text and binary data; optimized for sequential read/write operations.
    \item \textbf{Append Blobs}: Similar to block blobs but optimized for append operations, suitable for logging.
    \item \textbf{Page Blobs}: Optimized for random read/write operations, mainly used for virtual hard disks (VHDs).
\end{itemize}

\section{Creating a Storage Account and Containers}
Use the Azure Portal, Azure CLI, or ARM templates to create a storage account and blob containers.

\textbf{Azure CLI Commands:}
\begin{lstlisting}[language=bash]
# Create a Resource Group
az group create --name MyResourceGroup --location eastus

# Create a Storage Account
az storage account create --name mystorageaccount --resource-group MyResourceGroup --location eastus --sku Standard_LRS

# Create a Blob Container
az storage container create --name mycontainer --account-name mystorageaccount
\end{lstlisting}

\section{Working with Blobs}
Upload, download, and manage blobs using Azure Portal, Azure CLI, SDKs, or REST API.

\textbf{Hands-On Tip:}
\begin{itemize}
    \item Use Azure Storage Explorer to upload, download, and manage blobs visually.
    \item Practice using the Azure SDK for your preferred language (Python, .NET, Node.js) to perform CRUD operations on blobs.
\end{itemize}

\textbf{Azure CLI Commands:}
\begin{lstlisting}[language=bash]
# Upload a file to a blob container
az storage blob upload --account-name mystorageaccount --container-name mycontainer --name myfile.txt --file /path/to/local/file.txt

# Download a blob
az storage blob download --account-name mystorageaccount --container-name mycontainer --name myfile.txt --file /path/to/download/file.txt
\end{lstlisting}

\section{Managing Blob Access}
\begin{itemize}
    \item \textbf{Public Access Levels}: Configure containers to allow public access at the blob or container level.
    \item \textbf{Shared Access Signature (SAS)}: Generate temporary access tokens with specific permissions.
\end{itemize}

\textbf{Hands-On Tip:}
\begin{itemize}
    \item Create a SAS token with read permissions and test accessing a blob using the generated URL.
\end{itemize}

\textbf{Azure CLI Commands:}
\begin{lstlisting}[language=bash]
# Generate a SAS token for the storage account
az storage account generate-sas --account-name mystorageaccount --permissions r --resource-types o --services b --expiry 2024-12-31T23:59:59Z
\end{lstlisting}

\section{Blob Storage Security}
\begin{itemize}
    \item \textbf{Encryption}: Data is encrypted at rest and in transit using server-side encryption with customer-managed or Microsoft-managed keys.
    \item \textbf{Firewalls and Private Endpoints}: Secure access to your storage account using VNet rules, IP restrictions, and private endpoints.
\end{itemize}

\textbf{Hands-On Tip:}
\begin{itemize}
    \item Enable a private endpoint and test accessing the storage account from within a virtual network.
\end{itemize}

\section{Monitoring and Troubleshooting}
\begin{itemize}
    \item \textbf{Azure Monitor and Metrics}: Use Azure Monitor to track metrics like ingress/egress, availability, and capacity.
    \item \textbf{Diagnostics Logs}: Enable diagnostic settings to capture logs for requests, errors, and performance.
\end{itemize}

\textbf{Azure CLI Commands:}
\begin{lstlisting}[language=bash]
# Enable logging for blob storage
az storage logging update --account-name mystorageaccount --log rwd --retention 7 --services b
\end{lstlisting}

\section{Data Lifecycle Management}
Use Lifecycle Management policies to automatically transition blobs to cooler storage tiers (Hot, Cool, Archive) or delete them based on conditions.

\textbf{Hands-On Tip:}
\begin{itemize}
    \item Create a lifecycle policy to move blobs to the archive tier after 30 days of inactivity.
\end{itemize}

\end{document}
