
\documentclass{article}
\usepackage{geometry}
\geometry{a4paper, margin=1in}
\usepackage{hyperref}
\usepackage{titlesec}
\usepackage{listings}
\usepackage{xcolor}

\titleformat{\section}{\large\bfseries}{\thesection}{1em}{}

\title{AZ-204 Study Guide: Implement User Authentication and Authorization}
\author{}
\date{}

\lstset{
    basicstyle=\ttfamily\footnotesize,
    keywordstyle=\color{blue},
    commentstyle=\color{green},
    stringstyle=\color{red},
    breaklines=true,
    showstringspaces=false,
    frame=single,
    backgroundcolor=\color{gray!10}
}

\begin{document}

\maketitle

\section{Overview of Authentication and Authorization}
\begin{itemize}
    \item \textbf{Authentication}: The process of verifying the identity of a user or service. Common methods include passwords, tokens, certificates, and biometric factors.
    \item \textbf{Authorization}: The process of determining what an authenticated user or service is allowed to do, often managed via roles and permissions.
\end{itemize}

\section{Azure Active Directory (Azure AD)}
Azure AD is Microsoft’s cloud-based identity and access management service. Supports Single Sign-On (SSO), Multi-Factor Authentication (MFA), and integration with various identity providers (Google, Facebook, etc.).

\textbf{Key Concepts:}
\begin{itemize}
    \item \textbf{App Registrations}: Register applications with Azure AD to authenticate users and obtain access tokens.
    \item \textbf{Roles and Permissions}: Define roles to control access to Azure resources using Role-Based Access Control (RBAC).
\end{itemize}

\textbf{Azure CLI Commands:}
\begin{lstlisting}[language=bash]
# Create an App Registration
az ad app create --display-name "MyApp" --identifier-uris "https://myapp.example.com"

# Assign a role to a user
az role assignment create --assignee user@example.com --role Reader --scope /subscriptions/<subscription-id>/resourceGroups/MyResourceGroup
\end{lstlisting}

\section{Implementing OAuth 2.0 and OpenID Connect}
Use OAuth 2.0 for authorization and OpenID Connect for user authentication. Common use cases include enabling SSO, securing APIs, and protecting applications.

\textbf{Hands-On Tip:}
\begin{itemize}
    \item Set up OAuth 2.0 authentication for a web app by configuring it to use Azure AD as the identity provider.
\end{itemize}

\section{Microsoft Identity Platform}
The Microsoft Identity Platform is an evolution of Azure AD that provides developers with identity and authentication solutions. Integrates with OAuth 2.0, OpenID Connect, and Microsoft Graph API.

\textbf{Hands-On Tip:}
\begin{itemize}
    \item Use Microsoft Identity libraries (MSAL) for authentication in .NET, JavaScript, Python, and other languages.
\end{itemize}

\section{Implementing Azure AD B2C (Business-to-Consumer)}
Azure AD B2C allows external users (customers) to authenticate with your applications using their preferred identity provider (e.g., Google, Facebook, or local accounts). Supports customizable user flows for sign-in, sign-up, and password reset.

\textbf{Azure CLI Commands:}
\begin{lstlisting}[language=bash]
# Create an Azure AD B2C Tenant
az ad b2c create --display-name "MyB2CTenant" --domain "myb2ctenant.onmicrosoft.com"

# Register an Application in Azure AD B2C
az ad app create --display-name "MyB2CApp" --reply-urls "https://myapp.example.com"
\end{lstlisting}

\textbf{Hands-On Tip:}
\begin{itemize}
    \item Set up Azure AD B2C and configure custom policies for a personalized user experience.
\end{itemize}

\section{Secure Azure App Service with Authentication and Authorization}
Enable built-in authentication/authorization in Azure App Service to secure web apps and APIs. Integrate with Azure AD, Google, Facebook, or other identity providers.

\textbf{Azure CLI Commands:}
\begin{lstlisting}[language=bash]
# Enable Authentication on Azure App Service
az webapp auth update --resource-group MyResourceGroup --name MyAppService --enabled true --action LoginWithAzureActiveDirectory
\end{lstlisting}

\textbf{Hands-On Tip:}
\begin{itemize}
    \item Enable authentication on an Azure Web App using the Azure Portal and test access with various identity providers.
\end{itemize}

\section{API Management and OAuth 2.0}
Secure APIs using Azure API Management by integrating OAuth 2.0, OpenID Connect, or client certificates. Define API policies to enforce authentication and authorization rules.

\textbf{Hands-On Tip:}
\begin{itemize}
    \item Implement OAuth 2.0 security in an Azure API Management instance and test access control using tokens.
\end{itemize}

\section{Managed Identity for Azure Resources}
Managed Identity provides an automatically managed identity for Azure services to access other Azure resources securely. No need to manage credentials; identities can be used for authenticating to Azure Key Vault, Storage, etc.

\textbf{Azure CLI Commands:}
\begin{lstlisting}[language=bash]
# Enable System-Assigned Managed Identity for an App Service
az webapp identity assign --resource-group MyResourceGroup --name MyAppService

# Grant access to a Key Vault using Managed Identity
az keyvault set-policy --name MyKeyVault --object-id <identity-object-id> --secret-permissions get
\end{lstlisting}

\textbf{Hands-On Tip:}
\begin{itemize}
    \item Configure Managed Identity for an Azure Function and use it to securely access secrets from Azure Key Vault.
\end{itemize}

\section{Multi-Factor Authentication (MFA)}
MFA requires more than one form of verification, adding an extra layer of security. Azure AD supports MFA via phone call, text message, and authenticator app.

\textbf{Hands-On Tip:}
\begin{itemize}
    \item Enable MFA for a test user in Azure AD and configure verification options.
\end{itemize}

\end{document}
