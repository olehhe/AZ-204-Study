
\documentclass{article}
\usepackage{geometry}
\geometry{a4paper, margin=1in}
\usepackage{hyperref}
\usepackage{titlesec}
\usepackage{listings}
\usepackage{xcolor}

\titleformat{\section}{\large\bfseries}{\thesection}{1em}{}

\title{AZ-204 Study Guide: Implement Azure App Services Web Apps}
\author{}
\date{}

\lstset{
    basicstyle=\ttfamily\footnotesize,
    keywordstyle=\color{blue},
    commentstyle=\color{green},
    stringstyle=\color{red},
    breaklines=true,
    showstringspaces=false,
    frame=single,
    backgroundcolor=\color{gray!10}
}

\begin{document}

\maketitle

\section{Overview of Azure App Services}
Azure App Service is a fully managed platform for building, deploying, and scaling web apps. It supports multiple languages including .NET, Java, Node.js, Python, and PHP. It integrates with CI/CD pipelines, custom domains, and SSL certificates.

\section{Creating an Azure App Service Web App}
\begin{itemize}
    \item Use Azure Portal, Azure CLI, PowerShell, or ARM templates.
    \item Key components to configure: App Name, Runtime Stack, Region, and Pricing Plan.
    \item Pricing tiers (Basic, Standard, Premium) determine the resources available to your app.
\end{itemize}

\textbf{Azure CLI Commands:}
\begin{lstlisting}[language=bash]
# Create a Resource Group
az group create --name MyResourceGroup --location eastus

# Create an App Service Plan
az appservice plan create --name MyAppServicePlan --resource-group MyResourceGroup --sku B1

# Create a Web App
az webapp create --name MyUniqueAppName --plan MyAppServicePlan --resource-group MyResourceGroup
\end{lstlisting}

\section{Deployment Options}
\begin{itemize}
    \item \textbf{Continuous Deployment}: Integrates with GitHub, Azure Repos, Bitbucket, etc.
    \item Local Git, FTP, or ZIP file deployment for manual processes.
    \item Azure DevOps or GitHub Actions for CI/CD pipelines.
\end{itemize}

\textbf{Hands-On Tip:}
\begin{itemize}
    \item Set up a GitHub Actions workflow to automatically deploy your web app to Azure when code is pushed to the main branch.
\end{itemize}

\textbf{Azure CLI Commands:}
\begin{lstlisting}[language=bash]
# Deploy using ZIP file
az webapp deployment source config-zip --resource-group MyResourceGroup --name MyUniqueAppName --src path/to/deploy.zip
\end{lstlisting}

\section{Configuration and Settings}
\begin{itemize}
    \item \textbf{Application Settings}: Manage environment variables, connection strings.
    \item \textbf{Diagnostic Settings}: Enable application logging, web server logging, and monitoring.
    \item \textbf{Custom Domains \& SSL}: Configure custom domains and secure your app with SSL certificates.
\end{itemize}

\textbf{Hands-On Tip:}
\begin{itemize}
    \item Practice configuring custom domains and uploading SSL certificates in the Azure portal.
\end{itemize}

\section{Scaling and Performance Management}
\begin{itemize}
    \item \textbf{Vertical Scaling}: Adjust App Service Plan tier for more resources (CPU, RAM).
    \item \textbf{Horizontal Scaling}: Use autoscaling to manage instances based on demand (rules based on CPU, memory, etc.).
    \item \textbf{Traffic Routing}: Use Azure Traffic Manager for high availability and load balancing.
\end{itemize}

\textbf{Azure CLI Commands:}
\begin{lstlisting}[language=bash]
# Scale up the App Service Plan
az appservice plan update --name MyAppServicePlan --resource-group MyResourceGroup --sku P1v2

# Configure autoscaling rules
az monitor autoscale create --resource-group MyResourceGroup --resource MyAppServicePlan --min-count 1 --max-count 3 --count 1
\end{lstlisting}

\section{Deployment Slots}
Create deployment slots (e.g., staging, testing) to isolate code changes before production. Swap slots to promote code between environments without downtime.

\textbf{Hands-On Tip:}
\begin{itemize}
    \item Create a staging slot, deploy an updated version of your app, and then swap it with production to practice zero-downtime deployment.
\end{itemize}

\textbf{Azure CLI Commands:}
\begin{lstlisting}[language=bash]
# Create a deployment slot
az webapp deployment slot create --name MyUniqueAppName --resource-group MyResourceGroup --slot staging
\end{lstlisting}

\section{Security}
\begin{itemize}
    \item \textbf{Authentication/Authorization}: Integrate with Azure AD, Facebook, Google, etc.
    \item \textbf{Managed Service Identity (MSI)}: Securely connect to Azure services without managing credentials.
    \item \textbf{VNet Integration}: Connect securely to on-premises or Azure VNet resources.
\end{itemize}

\textbf{Hands-On Tip:}
\begin{itemize}
    \item Enable authentication with Azure AD in the portal and test access to the app.
\end{itemize}

\section{Monitoring and Troubleshooting}
\begin{itemize}
    \item \textbf{Application Insights}: Monitor performance, detect issues, and analyze usage.
    \item \textbf{Log Analytics}: Collect and analyze logs for in-depth troubleshooting.
\end{itemize}

\textbf{Azure CLI Commands:}
\begin{lstlisting}[language=bash]
# Enable Application Insights
az webapp log config --name MyUniqueAppName --resource-group MyResourceGroup --application-logging filesystem
\end{lstlisting}

\end{document}
